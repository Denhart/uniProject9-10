\section{Finite Difference Time Domain}
\label{sec:fdtd}
This section will describe the finite difference time domain method. This is a method for solving Maxwell equations in time domain. A brief introduction to Marxwell's Equations is given, in order to understand the methodology of FDTD. Then the teqniques and ideas behind FDTD are presenten and used on the 1D scalar wave-equation. Then an introduction to the Yee Algorithm, Yee cell and Yee notation is presenten and applied on the 3D Maxwell curl equation. Lastly the key parameters of FDTD are discussed and related to CST Microwavestudio.

\subsection{Introduction to Maxwell's Equations}
Maxwell's equations are a set of equations that describe how electric and magnetic fields propagate. The set of equations consists of Gauss' law, Gauss' Magnetism law, Faraday's law and finally Ampere's law. 

\subsubsection{Gauss' Law}
Gauss' law describes how the electric field behaves around electric charges. Below is Gauss' Law written in point differential-form, the equation states that the divergence of the electric flux density $\bold{D}$ is equal to the volume electric charge density $\rho_V$ \cite{taflove2000computional}. This is that the field around a point $\bold{D}$ is equal to electric charge density $\rho_V$. From this it is found that if there exists an electric charge, the divergence of $\bold{D}$ is nonzero, and only zero when there is no charge present. 

\begin{align}
\nabla \cdot \bold{D} &= \rho_V
\end{align}

\subsubsection{Gauss' Magnetism Law}
Gauss' magnetism law, is much like the previous formula, but for magnetic fields. Basically Gauss' law for magnetism states that a magnetic charge does not exist. This is given in the Equation below \cite{taflove2000computional}. 
\begin{align}
\nabla \cdot \bold{B} &=0
\end{align}
From this it is clear, that there are no magnetic monopoles, the divergence of $\bold{B}$ or $\bold{H}$ fields is always zero and that magnetic fields always flows in closed loops. 

\subsubsection{Faraday's Law}
Faraday's law states that the curl of the $\bold{E}$, is equal to the rate of change of a magnetic field. This is very important in the way that electric ways propagate. This is given in the Equation below \cite{taflove2000computional}. 
\begin{align}
\nabla \times \bold{E} &= - \frac{\partial \bold{B}}{\partial t}
\end{align}

From this equation, it can be seen that a time changing magnetic field gives rise to an E-field circulating around it. Likewise, a circulating E-field causes a time changing magnetic field.   

\subsubsection{Ampere's Law}
Ampere's law, is much like Faraday's law, but for curling magnetic fields. In simple words the Equation below stats that the curl of a magnetic field $\bold{H}$ is given by the rate of change of a electric field $\bold{D}$ termed displacement current density and the electric current density\cite{taflove2000computional}.

\begin{align}
\nabla \times \bold{H} &= \frac{\partial \bold{D}}{\partial t} + \bold{J} 
\end{align}
This is very much symmetric to Faraday's law, with an additional term. However the outcome is the same. That is that a time changing electric field gives rise to an H-field circulating around it and likewise a circulating H-field causes a time changing electric field. From Faraday and Ampere's law it is seen how waves can actually propagate, since a change in an electric field causes a curling magnetic field and so on.    

\subsubsection{Constitutive relations}
Before doing any mathematical manipulations or calculations, it is necessary to have an idea on how the different terms and variables are connected through physical constants. E.g. how the electric flux density is connected to the electric field, which can lead to simplification of calculations later on.

\paragraph{Electric Flux Density:} The electric flux density $\bold{D}$ is related to the electric field $\bold{E}$ by the permittivity measured in Farads per meter. Permittivity is a fundamental parameter of a given material, which affects the propagation of an electric field. Typically it is denoted by $\epsilon$. The relation is given by\cite{taflove2000computional}: 

\begin{align}
  \bold{D} = \epsilon \bold{E}
\end{align}

\paragraph{Magnetic Flux Density:} The magnetic flux density $\bold{B}$ is related to the magnetic field $\bold{H}$ by the permeability measured in Henries per meter. Permeability is a fundamental parameter of a given material, which affects the propagation of a magnetic field. Typically it is denoted by $\mu$. The relation is given by\cite{taflove2000computional}: 

\begin{align}
  \bold{B} = \mu \bold{H}
\end{align}
\paragraph{Electric Current Density:}  The electric current density $\bold{J}$ is related to the electric field $\bold{E}$ by the conductivity measured in Siemens per meter. Permeability is a fundamental parameter of a given material, which affects the current flow in a conductor. Typically it is denoted by $\sigma$. The relation is given by\cite{taflove2000computional}: 

\begin{align}
  \bold{J} = \sigma \bold{E}
\end{align}

\subsection{The 1D Wave Equation}
The wave equation is one of the most basic partial differential equation, which describes how waves propagate. This section will use the 1-dimension wave to describe the method of FDTD. Furthermore it is assumed that

\begin{itemize}
\item The region is free of charge and current: $\bold{J} = \bold{M} = 0$
\item The medium is linear (field-independent)
\item The medium is isotropic (direction-independent)
\item The medium is non-dispersive (frequency-independent)
\end{itemize}
Given these assumptions, Ampere's and Faraday's law can be rewritten as

\begin{align}
  \frac{\partial \epsilon \bold{E}}{\partial t} = \nabla \times \bold{H} - 0 & \implies \frac{\partial \bold{E}}{\partial t} = \frac{1}{\epsilon} \nabla \times \bold{H} \\
\frac{\partial \mu \bold{H}}{\partial t} =  - \nabla \times \bold{E} - 0 & \implies \frac{\partial \bold{H}}{\partial t} = - \frac{1}{\mu} \nabla \times \bold{E}
\end{align}
Expanding E- and H-field into the Cartesian components 

\begin{align}
  \frac{\partial E_x}{\partial t} = \frac{1}{\epsilon} \big(\frac{\partial H_z}{\partial y} - \frac{\partial H_y}{\partial t} \big) \qquad  
  \frac{\partial H_x}{\partial t} = \frac{1}{\mu} \big(\frac{\partial E_z}{\partial y} - \frac{\partial E_y}{\partial t} \big)\\
  \frac{\partial E_y}{\partial t} = \frac{1}{\epsilon} \big(\frac{\partial H_x}{\partial y} - \frac{\partial H_z}{\partial t} \big) \qquad 
  \frac{\partial H_y}{\partial t} = \frac{1}{\mu} \big(\frac{\partial E_x}{\partial y} - \frac{\partial E_z}{\partial t} \big) \\
  \frac{\partial E_z}{\partial t} = \frac{1}{\epsilon} \big(\frac{\partial H_y}{\partial y} - \frac{\partial H_x}{\partial t} \big) \qquad 
  \frac{\partial H_z}{\partial t} = \frac{1}{\mu} \big(\frac{\partial E_y}{\partial y} - \frac{\partial E_x}{\partial t} \big)
\end{align}
which for the 1D case $\frac{\partial}{\partial z} = \frac{\partial}{\partial y} = 0$ simplifies to 

\begin{align}
  \frac{\partial E_x}{\partial t} &= 0 \\
  \frac{\partial E_y}{\partial t} &= - \frac{1}{\epsilon} \frac{\partial H_z}{\partial x}\\
  \frac{\partial E_z}{\partial t} &= \frac{1}{\epsilon} \frac{\partial H_y}{\partial x}
\end{align}
The same steps can be used to get an expression for the H-field. For each direction we can combine the E and H expressions to get the 1D scalar wave equation, e.g. the z-direction\cite{taflove2000computional}: 

\begin{align}
  \frac{\partial^2 E_z}{\partial t^2} = c^2 \frac{\partial^2 E_z}{\partial x^2}
\end{align}
Where $c$ is the speed of light. This equation tells us that an electric field moving in the z-direction, will propagate along the x-axis in time, with a speed of $c$. Going back to the general wave-equation\cite{taflove2000computional}

\begin{align}
  \frac{\partial^2 u}{\partial t^2} = c^2 \frac{\partial^2 u}{\partial x^2}
\end{align}
Since the wave-equation is a second order partial differential equation, it is known to have two linearly independent solutions\cite{taflove2000computional}: 

\begin{align}
  u(x,t) = F(x+ct) + G(x-ct)
\end{align}
Both $F$ and $G$ are known as propagating-wave solutions, where an argument of the form $x+ct$ is a wave moving towards decreasing $x$ and an argument of the form $x-ct$ is a wave moving towards increasing $x$\cite{taflove2000computional}. In order to compute this numerically finite differences are used. Thus the following needs to be approximated using finite-differences 

\begin{align}
  \frac{\partial^2 u(x,t)}{\partial t^2},\frac{\partial^2 u(x,t)}{\partial x^2}
\end{align}

Taylor series expansion is used to approximate the expression at a point $x_i$ using finite differences: $u(x_i) = u(x_i+\Delta x) + u(x_i - \Delta x)$ which then is used to solve the wave-equation by substituting the two central differences into the 1D wave equation. This gives the solution for the latest time step as\cite{taflove2000computional}

\begin{align*}
  u_i^{n+1} =& \left( \frac{c\Delta t}{\Delta x} \right) \left[ u_{i+1}^n - 2u_i^n + u_{i-1}^n \right] + 2u_i^n -u_i^{n-1} \\
             &+ O[(\Delta t)^2] + O[(\Delta x)^2]
\end{align*}

\subsection{Dispersion}
When ever discrete numerical methods are used dispersion must be taken into account. Dispersion in this case can be seen as a variation in wavelength. For the wave equation this depends on the chosen time and space steps\cite{taflove2000computional}. 

\begin{itemize}
\item \textbf{Magic time-step:} $c\Delta t = \Delta x$. No dispersion.
\item \textbf{Very fine mesh:} $\Delta t \rightarrow 0$, $\Delta x \rightarrow 0$. No dispersion.
\item \textbf{Difference 1:} $c\Delta t < \Delta x$. Wavelength smaller in sampled space than in reality.
\item \textbf{Difference 2:} $c\Delta t > \Delta x$. Unstable as $\omega$ is complex. Exponential growth.
\end{itemize}

Generally the magic time-step is only an exact solution in 1D, but it exists for all dimensions but only for one incident angle\cite{taflove2000computional}. It is also seen that finer mesh cells reduces the error term.  

\subsection{The Yee Algorithm}
The Yee algorithm is a set of finite difference equations for Maxwell's curl equation system derived by Kane Yee. This algorithm solves both E- and H-fields in space and time using coupled curl equations instead of solving for one field at a time using the wave equation. Some of the key features of the Yee Algorithm is\cite{taflove2000computional}

\begin{itemize}
\item Both E- and H-field boundaries can be used
\item Solving both E- and H-fields is more robust 
\item Electric and magnetic material properties can be modeled easily
\item Unique field features, e.g. singularities. 
\end{itemize}
The Yee algorithm centers the E- and H-fields in 3D, such that every E-field is surrounded by four curling H-fields and every H-field is surrounded by four curling E-fields. This is illustrated on Figure~\ref{fig:fdtd-grid}, each cell referred to as the Yee-cell, a single Yee cell is shown in Figure~\ref{fig:yee-cell}. The arrangement of the electric and magnetic fields implicitly enforce Gauss' Law and Gauss' Magnetic Law, which makes the Yee mesh divergence free with respect to the E- and H-fields\cite{taflove2000computional}.  

\begin{figure}
    \centering
    \begin{subfigure}[b]{0.49\textwidth}
      \centering
        \includegraphics[scale=0.3]{img/analysis/fdtd_mesh}
        \caption{A Yee FDTD grid}
        \label{fig:fdtd-grid}
    \end{subfigure}
    ~
    \begin{subfigure}[b]{0.49\textwidth}
      \centering
        \includegraphics[scale=0.7]{img/analysis/yee_3d}
        \caption{A single Yee cell}
        \label{fig:yee-cell}
    \end{subfigure}
    \caption{Spatial interpetation of the Yee Algorithm}
    \label{fig:space-yee-algorithm}
\end{figure}

The centering of the E- and H-fields in time, is often termed the leapfrog method. This time stepping is fully explicit, and thus avoiding problems with simultaneous equations and matrix inversion. This leapfrog method is shown on Figure~\ref{fig:leap_frog} 

\begin{figure}[htbp]
    \centering
    \includegraphics[scale=0.7]{img/analysis/leap_frog}
    \caption{Leap-frog method illustrated on the time line}
    \label{fig:leap_frog}
\end{figure}

To keep an overview Yee, created the following notation which is useful when considering 3D FDTD equations

\begin{align}
  u_X (i \Delta x , j \Delta y, k \Delta z, n \Delta t) = u^n_{X_{i,j,k}}
\end{align}

%http://www.ualr.edu/wirelesslab/fdtd/fdtd3.pdf
\subsubsection{Finite Difference of Maxwell's Equations in 3D}
For completeness the previous ideas and notation can now be used to derive a numerical approximation of Maxwell's Curl Equations in 3D. The derivations will not be shown here, and only the solution for a single Cartesian component will be given here, due to its complexity. As an example the approximation for $E_x$ is shown below\cite{taflove2000computional}: 

\begin{align}
  &E_x |^{n+1/2}_{i,j+1/2,k+1/2} = \big( \frac{1-\frac{\sigma_{i,j+1/2,k+1/2} \Delta t}{2\epsilon_{i,j+1/2,k+1/2}}}{1+\frac{\sigma_{i,j+1/2,k+1/2} \Delta t}{2\epsilon_{i,j+1/2,k+1/2}}} \big) E_x |^{n-1/2}_{i,j+1/2,k+1/2}  + \\ &\big(  \frac{\frac{\Delta t}{\epsilon_{i,j+1/2,k+1/2}}}{1+\frac{\sigma_{i,j+1/2,k+1/2} \Delta t}{2\epsilon_{i,j+1/2,k+1/2}} }    \big) \big( \frac{H_z |^n_{i,j+1,k+1/2} - H_z |^n_{i,j,k+1/2} }{\Delta y} - \frac{H_y |^n_{i,j+1/2,k+1} - H_y |^n_{i,j+1/2,k} }{\Delta z} -J_{source}|^n_{i,j+1/2,k+1/2}  \big)
\end{align}

As seen the equations quickly become large and hard to handle, however to fully extend this to 3D this would have to be done for all the the Cartesian coordinates for both E- and H-fields. 

\subsection{Absorbing Boundary Conditions}
Since many geometries of interest are defined in open regions, e.g. antennas, where the spatial domain is unbounded. Since computers have limited storage, it is needed to find a suitable boundary condition on the outer perimeter of the domain $\Omega$ in order to simulate to infinity. Thus it is needed to find a boundary condition that allows outward propagating numerical wave to exit the outer perimeter of $\Omega$, without spurious reflections from the outgoing waves. 

One way of achieving the absorbing boundary condition is by terminating the outer boundary with an absorbing material, which is analogous to the method used in anechoic chambers. For this perfectly matched layer (PML) is used, which is derived by Berenger. This allows plane waves of arbitrary incidence, polarization and frequency to be matched at the boundary. PML usually gives a back-reflection in the order of $\approx 10^{-6}$--$10^{-8}$.\cite{taflove2000computional}

\subsection{FDTD Parameters and Relation to CST}
For all simulations in this report CST microwave studio is being used. The following will try to connect the FDTD theory to the parameters in CST. Even though CST uses Finite Integration Technique (FIT) instead of finite differences. FIT share the structure of FDTD, but using the Maxwell's Equations on integral form. The advantage of FIT is that is uses less memory and it is allows for easier code implementation of some features. Besides this most of the basic parameters are the same. 

\subsubsection{Cell Size and Meshing}
The coice of cell size is very important when doing FDTD simulations. A general rule of thumb is that a cell should be much less than the smallest wavelength, it is often said that 10 cells per wavelength is sufficient, that is the cell size should be $\lambda/10$. In advanced FDTD simulations an adaptive meshing is often used, where even smaller cells are used in dense materials with high field locations, and then large cells outside these areas. When doing this it is also very important not to change the cell size too much compared to the neighbor cells. It is also important
that the cell size are small enough to approximate the geometry which is to be modeled accurately. Geometries such as circles are being approximated with rectangular cells, this causes a ``staircase effect'' on the object, and the representation can be inaccurate if the cell sizes used are too large. 

Another rule of thumb, which can help assessing the right cell size is that the smallest mechanical item of the structure is to be approximated by at least two cells. The rule of thumb says that\cite{kunz1993fdtd}:
\begin{align}
    \Delta = \frac{d_{\text{min}}}{N_d} 
\end{align}
where 
\begin{where}
\item[$\Delta$] Is the length/width/height of the cell.
\item[$d_{\text{min}}$] Smallest physical dimension.
\item[$N_d$] Oversampling factor -- should be $\geq 1$.
\end{where}


\subsubsection{Time Step Size}
The time step size is another parameter that has to be chosen with care, when doing FDTD simulations. The time step should be small enough, such that at any point a wave should not pass through more than one cell.  Calculating the time step is dependent on the cell size and dimensions and the propagation speed.  The time step for the 3D rectangular grid can be calculated as:
\begin{align}
   \Delta t &= \frac{1}{c \sqrt{\frac{1}{(\Delta x)^2}+\frac{1}{(\Delta y)^2}+\frac{1}{(\Delta z)^2}}} \\
            &= \frac{1}{c \sqrt{\frac{3}{\Delta^2}}} \\
            &= \frac{\Delta}{c \sqrt{3}} \label{eq:deltat}           
\end{align}
where:
\begin{where}
\item [$c$] Speed of light.
\item [$\Delta = \Delta x = \Delta y = \Delta z$] Size of the cell side for rectangular cell.
\end{where}
The step size should also be reduced for the materials with conductivity higher then zero. If the time step is chosen to be larger than given in Equation~\ref{eq:deltat} instabilities can occur \cite{kunz1993fdtd}.   

\subsubsection{Accuracy and time limit}
These terms are taking from CST microwave studio, and are figures that limit the simulation either when a certain accuracy criteria has been met or when it has been simulation for a certain amount of time. In practice it is not wanted that a simulation is to be stopped due to a given time limit, therefor this limit is often set to a very large number. In CST accuracy is a figure that describes how much energy is left in the system. The accuracy setting stops the simulation when a certain energy level within the structure has decreased to a chosen level. 
