\subsection{Example LTE Link Budget}
\begin{aautop}
In this section, an example link budget is given to show what parts of the link is affected by the antenna design. The link shows the total path loss expected in a link. Using the path loss models from Section~\ref{sec:propmodels}, this path loss can be used to estimate the possible communication distance.
\end{aautop}

The presented link budget is for typical values and is largely based on \cite{holma2011lte}. The maximum path loss is computed as
\begin{equation}
    \label{eq:linkbudget}
    \begin{aligned}
        \text{max path loss} =& \text{power} + \text{gains}\\
        &- \text{SNR requirement} - \text{noise} - \text{losses} - \text{margin}
    \end{aligned}
\end{equation}
where all quantities are given in dB. 

The values for the link budget are shown in Table~\ref{tab:linkbudget}. The entries dependent on the antenna design are described below.

\paragraph{TX Antenna Gain} The antenna gain is defined as the product of the directivity and the total efficiency of an antenna \cite{balanis2012antenna}. However, for a MIMO system, the multiplexing efficiency is a better metric than total efficiency as it also takes the correlation between antennas into account (see Section~\ref{sec:muxefficiency}). % The antenna gains for the link budget are then found by summing the individual directivities and the multiplexing efficiency.
	
\paragraph{Body Loss} The body loss is caused partially by the user absorbing power and partially by the user detuning the antennas. The detuning may be helped using a digital tuner to retune the antenna in the presence of a user.

\def\MARK{$^{\dagger}$\xspace}
\begin{table}[htbp]
    \centering
    \begin{tabular}{|l|c|r|r|}
        \hline
        \textbf{Power}         &                     &        &     \\
        Maximum TX power       & \cite{holma2011lte} & 23     & dBm \\
        \hline
        %
        \textbf{Antenna gains} &                     &        &     \\
        TX antenna gain        & \MARK               & 0      & dBi \\
        RX antenna gain        & \cite{holma2011lte} & 18     & dBi \\
        \hline
        %
        \textbf{SNR}           &                     &        &     \\
        SNR requirement        & \cite{holma2011lte} & $-7$   & dB  \\
        \hline
        %
        \textbf{Noise}         &                     &        &     \\
        eNode B noise figure   & \cite{holma2011lte} & 2      & dB  \\
        Thermal noise          & \cite{holma2011lte} & $-119$ & dBm \\
        \hline
        %
        \textbf{Losses}        &                     &        &     \\
        Body loss              & \MARK               & 0      & dB  \\
        Cable loss             & \cite{holma2011lte} & 2      & dB  \\
        \hline
        %
        \textbf{Margins}       &                     &        &     \\
        Interference margin    & \cite{holma2011lte} & 2      & dB  \\
        \hline
        %
        \textbf{Maximum path loss (Equation~\ref{eq:linkbudget})} & & 161    & dB  \\
        \hline
    \end{tabular}
    \caption{Example uplink link budget for LTE. \MARK{}Dependent on/may be improved by the antenna design.}
    \label{tab:linkbudget}
\end{table}

It is desired to have as low a maximum path loss as possible. The antennas developed in the report will therefore aim to fulfill the following criteria:
\begin{itemize}
    \item Increase the antenna gains by increasing the total efficiency.
    \item Increase the effective antenna gain (multiplexing efficiency or diversity gain) by lowering the farfield correlation between the two antennas.
    \item Lower the effect of body loss by using a digitally tunable capacitor to re-tune the antenna in the presence of a user.
\end{itemize}
