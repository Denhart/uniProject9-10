\chapter{Test Specification}
\label{cha:testspec}

\begin{aautop}
    In this chapter, it will be described how the specific requirements from Chapter~\ref{cha:reqspec} will be tested.
\end{aautop}

\paragraph{Test of Requirement~\sreqref{fbands} -- Frequency Bands/Tunable Range}
The frequency bands are chosen according to the LTE bands \cite{radio2015electronics}. This specification determines what is ``in-band'' for other requirements, below, and is not tested by itself.

\paragraph{Test of Requirement~\sreqref{bandwidthlow} -- Minimum Tunable Bandwidth in the Low Band}
The return loss, correlation, and efficiency is measured within the low band for the capacitance range specified in Requirement~\sreqref{tunable}. The specified bandwidth must be obtained for a minimum of one capacitance value. See Requirements~\sreqref{retloss}, \sreqref{correlation}, and \sreqref{efficiency}.

\paragraph{Test of Requirement~\sreqref{bandwidthhigh} -- Minimum Tunable Bandwidth in the High Band}
The return loss, correlation, and efficiency is measured within the high band for the capacitance range specified in Requirement~\sreqref{tunable}. The specified bandwidth must be obtained for a minimum of one capacitance value. See Requirements~\sreqref{retloss}, \sreqref{correlation}, and \sreqref{efficiency}.

\paragraph{Test of Requirement~\sreqref{physdim} -- Physical Dimensions}
The physical dimensions of the PCB, antenna, and the external limitations are measured with a measuring tape. 

\paragraph{Test of Requirement~\sreqref{copper} -- Copper Thickness}
The PCB is chosen to have a copper thickness according to the requirement.

\paragraph{Test of Requirement~\sreqref{retloss} -- In-band Return Loss}
The return loss is determined by measuring the $S$-parameters on a vector network analyzer (VNA). The return loss is then $\text{RL}_1 = -|S_{11}|$ for the top antenna and $\text{RL}_2 = -|S_{22}|$ for the side antenna \cite{pozar2011microwave}.

\paragraph{Test of Requirement~\sreqref{correlation} -- In-band Correlation Between Antenna Elements}
The correlation between antennas is determined from the farfield pattern, measured in the Satimo StarLab, assuming isotropic distribution of incoming waves. The correlation is then found according to Equation~\ref{eq:envlop_corr}.

\paragraph{Test of Requirement~\sreqref{efficiency} -- In-band Total Efficiency in Free-Space}
The total efficiency is determined from the farfield pattern, measured in the Satimo StarLab.
The total efficiency is then found according to Equation~\ref{eq:mimo_total_eff_satimo}.

\paragraph{Test of Requirement~\sreqref{sar} -- Maximum SAR}
The maximum SAR is simulated in CST with a phone case, modeled around the PCB, and a PEC screen. The SAR is only simulated -- not measured.

\paragraph{Test of Requirement~\sreqref{tunable} -- Tunable Capacitor Range} 
The tunable capacitor is chosen according to this requirement and is not tested.

\paragraph{Test of Requirement~\sreqref{ltepower} -- Reference Power Level}
The specified power level is used for SAR simulations and is not tested.

\begin{aautail}
The test specifications will be used to evaluate the antennas designed in the next chapter, where three initial antennas are designed and simulated.
\end{aautail}


% \cite{cita2015}
