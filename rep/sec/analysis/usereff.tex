\section{User Effects}
\label{se:user_effects}
In this section the user impact on mobile antenna performance will be described, including both effects of the head, hand and the body in general.


Antenna parameters such as efficiency, radiation pattern, impedance etc. will be effected by the user, as the body of the user will look like a lossy and large dielectric body from the antennas point of view. 
The internal antennas that are implemented in every phone today, have a similar performance compared to external stubby antennas, which were used in a numerous of older designs. However this is only if the antennas are measured and compared in free space or next to a phantom head. In practical use a mobile phone with an internal antenna design, will be much more vulnerable to head and hand impacts from the user. The negative performance impact will be there, but will differ from user to user, as things like head and hand size, characteristics, hold position, left or right hand etc. varies from different users. This is of course a drawback in switching from external to internal antennas, but the development also comes with a lot of advantages. The internal antenna provides robust design and normally has higher performance in mechanical tests, such as drop tests, wearing tests etc. as the antenna is placed inside the phone, thus making physical interaction impossible. 
To counteract and minimize the user effect problem to the internal antennas, some basic design guidelines can be followed.

\subsection{Design Guidelines}
An antenna that is placed in the top of the phone, will be more effected by the user in talk mode, as the antenna will be closer to the head. To avoide this a ground plane can be placed between the user and the head in order to create more isolation. However placing the antenna on top of the ground plane decreases the bandwidth significantly, which leads to a, increase in the antenna size to keep the required bandwidth. This is not a reliable solution in mobile phones, as the size requirement of the antennas are very strict as a result of recent phone designs, with bigger screens and smaller cases. A way to solve this problem is to place the antenna in the bottom of the antenna, thus lowering the user effect and makes is possible to place the antenna without the ground plane as isolation. This solution provides a certain ground clearence, which makes it possible to decrease the volume and thickness of the antenna, thus saving place in the mobile phone. 
This design was proven to work by Motorola with there Motorola Razor V3 phone, which was the first phone to use the bottom placed antenna. There were some skeptics to this design, as the bottom of the phone would be placed in the middle of the hand in talk mode. However Motorola proved that theory wrong and therefore many phone manufactors are using there own bottom placed antennas. This solution of course only applies in talk mode, as the head effect is close to zero, when the phone is used in data mode. In data mode the phone is usually held either in horizontal mode with one hand or in vertical mode with two hands. In horizontal mode, given the size of today's smartphones, the hand will be placed around the middle of the phone, thus clearing the top and bottom. This indicates that it does not matter whether the antenna is placed in the top or bottom of the phone. In the case of horizontal mode with two hands, the hands covers both the top and the bottom of the phone, thus it still makes no difference if the antenna is placed in the bottom or in the top of the phone.   

\subsection{Recent Measurements}

\begin{itemize}
\item Gerts and Samanthas paper
\end{itemize}




 



