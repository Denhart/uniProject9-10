\subsubsection{ecc(Eth1, Eth2, Eph1, Eph2)}
Compute the envelope correlation coefficient between two farfields. The
farfields are split into $\theta$ and $\phi$ part. Each part is a matrix with
$\theta$ on one axis and $\phi$ on the other.

\begin{lstlisting}[numbers=none, frame=none, xleftmargin=0em, language=, basicstyle=\footnotesize\ttfamily]
- Eth1: E-field, theta part, antenna 1
- Eth2: E-field, theta part, antenna 2
- Eph1: E-field, phi part, antenna 1
- Eph1: E-field, phi part, antenna 2

Return:
Envelope Correlation Coefficient (scalar)

Note:
https://mns.ifn.et.tu-dresden.de/Lists/nPublications/Attachments/612/Wang_Q_WSA_10.pdf
\end{lstlisting}

\subsubsection{intsphere(r, theta, phi)}
Do a spherical integral of a $(\theta \times \phi)$ matrix.

\begin{lstlisting}[numbers=none, frame=none, xleftmargin=0em, language=, basicstyle=\footnotesize\ttfamily]
- r: Matrix to integrate (x-axis=phi, y-axis=theta).
- phi: Phi axis values.
- theta: Theta axis values.

Return:
Scalar result of the integration.
\end{lstlisting}

\subsubsection{plot3d(r, stride=1, th\_lim=(0, 180), ph\_lim=(0, 360))}
Plot a matrix, $(\theta \times \phi)$, in 3D space.

\begin{lstlisting}[numbers=none, frame=none, xleftmargin=0em, language=, basicstyle=\footnotesize\ttfamily]
- r: Matrix to plot.
- stride: Resolution of the output. 1=detailed+slow, 10=rough+fast.
- th_lim: Upper and lower theta limits (degrees).
- ph_lim: Upper and lower phi limits (degrees).
\end{lstlisting}

\subsubsection{plotflat(r, th\_lim=(0,180), ph\_lim=(0,360), cmap="jet")}
Plot a farfield-matrix as a color-map. Remember that \ang{0} is the bottom of
the plot in spherical coordinates.

\begin{lstlisting}[numbers=none, frame=none, xleftmargin=0em, language=, basicstyle=\footnotesize\ttfamily]
- r: Matrix to plot (theta x phi).
- th_lim: Minimum and maximum theta/y-axis value (degrees).
- ph_lim: Minimum and maximum phi/x-axis value (degrees).
- cmap: Color map to use.
\end{lstlisting}

