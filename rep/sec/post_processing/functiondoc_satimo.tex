\subsubsection{col2mat(column, ntheta=SATIMO\_NUM\_ELEVATION, nphi=SATIMO\_NUM\_AZIMUTH)}
Convert a Satimo PM-exported column to a matrix with phi the 
x-axis and theta on the y-axis.

\begin{lstlisting}[numbers=none, frame=none, xleftmargin=0em, language=, basicstyle=\footnotesize\ttfamily]
- column: Column from a Satimo PM export.
- ntheta: Number of rows in the output (theta in the input).
- nphi: Number of columns in the output (phi in the input).

Return:
Matrix with phi on the x-axis and theta on the y-axis.
\end{lstlisting}

\subsubsection{ecc(Eth1, Eth2, Eph1, Eph2)}
Compute the envelope correlation coefficient between two Satimo farfields.
The farfields are split into theta and phi part. Each part is a matrix with
theta on one axis and phi on the other. This function uses the
satimo.intsphere() and not l3d.intsphere() and is therefore more accurate for
Satimo measurements.

\begin{lstlisting}[numbers=none, frame=none, xleftmargin=0em, language=, basicstyle=\footnotesize\ttfamily]
- Eth1: List of E-fields, theta part, antenna 1
- Eth2: List of E-fields, theta part, antenna 2
- Eph1: List of E-fields, phi part, antenna 1
- Eph1: List of E-fields, phi part, antenna 2

Return:
Envelope Correlation Coefficients (array, one for each element in the
        input lists.)

Note:
https://mns.ifn.et.tu-dresden.de/Lists/nPublications/Attachments/612/Wang_Q_WSA_10.pdf
\end{lstlisting}

\subsubsection{efficiency(trxfile, calfiles=[], reffiles=[], f\_tot=[], P\_tot=[])}
Get the total efficiency of a trx file, exported from Satimo.

\begin{lstlisting}[numbers=none, frame=none, xleftmargin=0em, language=, basicstyle=\footnotesize\ttfamily]
- trxfile: File, exported from Satimo PM, to compute the efficiency of.
- calfiles: List of calibration measurement files (trx files).
- reffiles: List of reference files relating to the calibration
        measurements.
- f_tot: Frequency axis of an existing total-power table.
- P_tot: Power axis of an existing total-power table. If an existing
        total-power table is supplied, the efficiency computation is much faster.

Return:
[f,eff] -- the total efficiency, eff, for each frequency, f.
\end{lstlisting}

\subsubsection{intsphere(E)}
Compute a spherical integral over the sphere recorded in Satimo. This is used
for efficiency and correlation computation. The integration is not over a
whole sphere, as a probe is missing at theta = 180 deg. Due to the low number
of samples recorded in Satimo, the integration must be done carefully, only
considering each sample one time. This is why l3d.intsphere() is not used.

\begin{lstlisting}[numbers=none, frame=none, xleftmargin=0em, language=, basicstyle=\footnotesize\ttfamily]
- E: Field matrix to integrate (theta x phi).

Return:
Spherical integral of E.
\end{lstlisting}

\subsubsection{intsphere\_alt(Etot)}
Alternative way of computing the radiated power/surface integral. For a small
sample size, this method is not as accurate as intsphere().

\begin{lstlisting}[numbers=none, frame=none, xleftmargin=0em, language=, basicstyle=\footnotesize\ttfamily]
- Etot: Matrix (theta x phi) from Satimo PM to compute the radiate power
        of (surface integration).

Return:
Surface integral of Etot ~ radiated power.
\end{lstlisting}

\subsubsection{loadref(f)}
Load a reference file containing S11, Gain, and Efficiency of a
reference/calibration antenna.
The reference files are cut to the following ranges, depending on file name:
\begin{itemize}
\item HomeRef600: No crop.
\item SD740-70: 700--800 MHz.
\item SD850-02: 800--900 MHz.
\item SD900-52: 900--1000 MHz.
\item SD1900-49: No crop.
\item SD2050-36: No crop.
\item SD2450-43: No crop.
\end{itemize}
Note that the 740 MHz reference file has the efficiency and gain columns
swapped!

\begin{lstlisting}[numbers=none, frame=none, xleftmargin=0em, language=, basicstyle=\footnotesize\ttfamily]
- f: File name.

Return:
M[0]=frequency(Hz), M[1]=S11(.), M[2]=Gain(.), M[3]=Eff(.)
\end{lstlisting}

\subsubsection{loadtrx(f)}
Load a trx measurement file from Satimo PM into memory. 

\begin{lstlisting}[numbers=none, frame=none, xleftmargin=0em, language=, basicstyle=\footnotesize\ttfamily]
- f: TRX file to load.

Return:
[f, list_horiz, list_vert] where
        f = [f1, f2, f3, ...]
        list_horiz = [E_horiz_f1, E_horiz_f2, E_horiz_f3, ...]
        list_vert  = [E_vert_f1,  E_vert_f2,  E_vert_f3,  ...]
and each "E" is a complex matrix, (theta x phi), containing the received fields.
\end{lstlisting}

\subsubsection{ma(values, window)}
Moving average filter.

\begin{lstlisting}[numbers=none, frame=none, xleftmargin=0em, language=, basicstyle=\footnotesize\ttfamily]
- values: Values/array to filter.
- windows: Size of the moving average filter.

Return:
Moving average filtered values.
\end{lstlisting}

\subsubsection{mat2col(mat, ntheta=SATIMO\_NUM\_ELEVATION, nphi=SATIMO\_NUM\_AZIMUTH)}
Convert a (theta x phi) matrix into the original column-format from the trx
file.

\begin{lstlisting}[numbers=none, frame=none, xleftmargin=0em, language=, basicstyle=\footnotesize\ttfamily]
- mat: Matrix with phi on the x-axis and theta on the y-axis.
- ntheta: Number of rows in the output (theta in the input).
- nphi: Number of columns in the output (phi in the input).

Return:
Column like a Satimo PM export, like the trx format.
\end{lstlisting}

\subsubsection{radiatedpower(h,v)}
Compute the radiated power for each frequency in the h and v list, using
radiatedpower\_single().

\begin{lstlisting}[numbers=none, frame=none, xleftmargin=0em, language=, basicstyle=\footnotesize\ttfamily]
- h: List of complex (theta x phi) matrices -- one for each frequency.
        Horizontal polarization.
- v: List of complex (theta x phi) matrices -- one for each frequency.
        Vertical polarization.

Return:
A vector with the radiated power for each frequency/element of h and
        v.
\end{lstlisting}

\subsubsection{totalpower\_table(calfiles, reffiles)}
Make a table of calibrated ``total power'' for each frequency.
Having (1) the total power and (2) the radiated power for a given antenna,
makes it possible to compute the total efficiency of the antenna.

\begin{lstlisting}[numbers=none, frame=none, xleftmargin=0em, language=, basicstyle=\footnotesize\ttfamily]
- calfiles: List of calibration measurement files (trx files) (order:
        lowest to highest frequency).
- reffiles: List of reference files relating to the calibration
        measurements (order: lowest to highest frequency).

Return:
[f,Ptot] -- the total power for each frequency.
\end{lstlisting}

