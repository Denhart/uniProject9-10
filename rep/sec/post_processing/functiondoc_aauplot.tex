\subsubsection{correlation(f, ecc, c="-", label="")}
Plot correlation

\begin{lstlisting}[numbers=none, frame=none, xleftmargin=0em, language=, basicstyle=\footnotesize\ttfamily]
- f: Frequency axis.
- ecc: Envelope correlation coefficient.
- c: Color/linetype string (e.g. '--b' for dashed blue).
- label: Label for the graph's legend.
\end{lstlisting}

\subsubsection{efficiency(f, e, c="-", label="")}
Plot an efficiency graph.

\begin{lstlisting}[numbers=none, frame=none, xleftmargin=0em, language=, basicstyle=\footnotesize\ttfamily]
- f: Frequency axis.
- e: Efficiency (. or dB).
- c: Color/linetype string (e.g. '--b' for dashed blue).
- label: Label for the graph's legend.
\end{lstlisting}

\subsubsection{end\_correlation(loc=1, fontsize=8)}
Finish the correlation plot with legend, etc.

\begin{lstlisting}[numbers=none, frame=none, xleftmargin=0em, language=, basicstyle=\footnotesize\ttfamily]
- loc: Location of the legend (like matplotlib.pyplot.legend())
- fontsize: Font size for the legend.
\end{lstlisting}

\subsubsection{end\_efficiency(**kwargs)}
Finish the efficiency plot with legend, etc.

\begin{lstlisting}[numbers=none, frame=none, xleftmargin=0em, language=, basicstyle=\footnotesize\ttfamily]
- kwargs: Arguments passed on to matplotlib.pyplot.legend()
\end{lstlisting}

\subsubsection{end\_sar(**kwargs)}
Finish the SAR plot with legend, etc.

\begin{lstlisting}[numbers=none, frame=none, xleftmargin=0em, language=, basicstyle=\footnotesize\ttfamily]
- kwargs: Arguments passed on to matplotlib.pyplot.legend()
\end{lstlisting}

\subsubsection{end\_sparam(**kwargs)}
Finish the $S$-parameter plot with legend, etc.

\begin{lstlisting}[numbers=none, frame=none, xleftmargin=0em, language=, basicstyle=\footnotesize\ttfamily]
- kwargs: Parameters passed onto matplotlib.pyplot.legend().
\end{lstlisting}

\subsubsection{figure(*args, **kwargs)}
Set up a figure of the correct dimensions and the correct font for the
report.

\begin{lstlisting}[numbers=none, frame=none, xleftmargin=0em, language=, basicstyle=\footnotesize\ttfamily]
- args: Positional arguments for matplotlib.pyplot.figure()
- kwargs: All arguments are passed onto the matplotlib.pyplot.figure()
        function.
\end{lstlisting}

\subsubsection{freqscale(f)}
Scale to get frequency axis to MHz

\begin{lstlisting}[numbers=none, frame=none, xleftmargin=0em, language=, basicstyle=\footnotesize\ttfamily]
- f: Frequency axis.
\end{lstlisting}

\subsubsection{sar(f, sar, c="-", label="")}
Plot a SAR graph.

\begin{lstlisting}[numbers=none, frame=none, xleftmargin=0em, language=, basicstyle=\footnotesize\ttfamily]
- f: Frequency axis.
- e: Efficiency (. or dB).
- c: Color/linetype string (e.g. '--b' for dashed blue).
- label: Label for the graph's legend.
\end{lstlisting}

\subsubsection{sparam(f, s, c="-", label="")}
Plot an $S$-parameter.

\begin{lstlisting}[numbers=none, frame=none, xleftmargin=0em, language=, basicstyle=\footnotesize\ttfamily]
- f: Frequency axis for the plot.
- s: S-parameter (abs-value in dB) to plot.
- c: Color/linetype string (e.g. '--b' for dashed blue).
- label: Label for the legend
\end{lstlisting}

\subsubsection{to\_db(x)}
Convert/preserve data in dB

\begin{lstlisting}[numbers=none, frame=none, xleftmargin=0em, language=, basicstyle=\footnotesize\ttfamily]
- x: Data to convert (e.g. efficiency).

Return:
Data in dB.
\end{lstlisting}

