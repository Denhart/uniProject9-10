\chapter{Preliminary Design and Simulation}
\label{cha:nousersim}
In this chapter, the design of three antenna designs will be documented. The antennas are developed from the requirement specification in Chapter~\ref{cha:reqspec}. The documentation will contain the following:
\begin{description}
\item[Technical drawing] A technical drawing with all dimensions of the antennas. 
\item[Circuit and component values] A schematic showing the matching circuit and component values.
\item[Description and surface currents] A description of how the antenna works and the surface currents are distributed at different frequencies.
\item[S-parameters] A figure of all S-parameters at the minimum capacitance values.
\item[S-parameter sweep] A sweep of S-parameters. For one antenna, the tunable capacitor is swept from \SI{0.3}{pF} to \SI{2.9}{pF} in \SI{0.2}{pF} steps. The other antenna has a fixed capacitance of \SI{0.3}{pF}. Four plots are shown
    \begin{enumerate}
    \item $S_{11}$, sweeping the capacitor of antenna 1 and keeping the capacitor for antenna 2 fixed.
    \item $S_{21}$, sweeping the capacitor of antenna 1 and keeping the capacitor for antenna 2 fixed.
    \item $S_{21}$, sweeping the capacitor of antenna 2 and keeping the capacitor for antenna 1 fixed.
    \item $S_{22}$, sweeping the capacitor of antenna 2 and keeping the capacitor for antenna 1 fixed.
    \end{enumerate}
\item[Tunable bandwidth] A table of the maximum obtained bandwidth for each antenna and for the low and the high band. The maximum bandwidth is found from the S-parameter sweep at the capacitor value yielding the best bandwidth.
\item[Correlation sweep] A sweep of the envelope correlation between the two antennas. One sweep for each antenna, keeping the capacitance value of the opposite antenna fixed at \SI{0.3}{pF}.
\item[Efficiency sweep] A sweep of the total efficiency of each antenna keeping the capacitance value of the opposite antenna fixed at \SI{0.3}{pF}.
\end{description}

The simulations in this chapter are performed in free space, i.e.\ with no user near the antennas.
