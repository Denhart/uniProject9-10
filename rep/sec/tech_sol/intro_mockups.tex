\chapter{Prototypes}
\label{cha:prototypes}

As the preliminary simulations have now been finished, the next step is to show the real-life performance of the three antenna designs. The goal is to see, first of all, how the antennas perform with respect to efficiency and return loss but also to see how stable the designs are as the real-life components are very different than the ideal components used in the simulations.

A prototype of each of the three designs have been built and the results are documented in this chapter. The prototypes each consist of a single-sided ground plane with the matching networks located as close to the antenna feeds as possible. Both the matching components and the tuning capacitors are implemented with 0402 and 0603 surface mount components to take up as little space on the board as possible while still being hand-solderable. The tuning is performed by simply replacing the tuning capacitor for each measurement.

The following values are used for the tuning capacitor sweep:
\begin{itemize}
\item \SI{0.3}{pF}
\item \SI{0.7}{pF}
\item \SI{1.1}{pF}
\item \SI{1.5}{pF}
\item \SI{2.0}{pF}
\item \SI{2.7}{pF}
\item \SI{3.0}{pF}
\end{itemize}
The antenna, which is not being swept, is terminated with a \SI{0.3}{pF} tuning capacitor like in the simulations in Chapter~\ref{cha:nousersim} and \ref{cha:usereff}.

\fixme{Changes: Mockup to prototype}
