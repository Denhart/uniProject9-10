
% vim: nospell
\documentclass[twoside,10pt]{report}

%   Packages
\usepackage[utf8]{inputenc}
\usepackage[T1]{fontenc}
\usepackage[english]{babel}
\usepackage[final]{pdfpages}
\usepackage[section]{placeins}
\usepackage[per-mode = symbol,detect-all=true]{siunitx}
\usepackage[draft,silent,nomargin,inline]{fixme}
\usepackage[compact,noindentafter]{titlesec}
\usepackage{subcaption}
\usepackage[font=footnotesize]{caption}
\usepackage{mathptmx}
\usepackage{courier}
\usepackage{amsmath}
\usepackage{amsfonts}
\usepackage{amssymb}
\usepackage{aurical}
\usepackage{booktabs}
\usepackage{calc}
\usepackage{commath}
\usepackage{pgfplots}
\usepackage{tabularx}
\usepackage{multicol}
\usepackage{graphicx}
\usepackage{gensymb}
\usepackage{url}
\usepackage{tikz}
\usepackage{listings}
\usepackage{hyperref}
\usepackage{longtable}
\usepackage{textcomp}
\usepackage{pdflscape}
\usepackage{float}
\usepackage{lastpage}
\usepackage{xspace}
\usepackage{wasysym}
\usepackage{pageslts}
\usepackage{setspace}
%matlab code font, and such
%   Package setup
\usetikzlibrary{shapes,arrows,positioning,patterns,decorations.markings,decorations.pathmorphing}
\captionsetup[subfigure]{font=footnotesize}
% \usetikzlibrary{external} 
% \tikzexternalize[prefix=img/tikz/]
\tikzset{
    >=latex,
}
\pgfplotsset{
    compat=newest,
    every axis plot/.append style = {
        very thick,
    },
    every axis legend/.append style={
        font={\scriptsize},
    },
    every axis/.append style={
        legend style={draw=none},
        legend cell align=left,
        axis x line=bottom,
        axis y line=left,
        scaled ticks=false,
        xticklabel = \pgfmathparse{\tick*1}\sisetup{scientific-notation = engineering}\num{\pgfmathresult},
        yticklabel = \pgfmathparse{\tick*1}\sisetup{scientific-notation = engineering}\num{\pgfmathresult},
        width=16cm,
        height=5cm,
        axis line style=-latex,
        tick scale binop=\times,
        /pgf/number format/.cd,
        set thousands separator={ },
        set decimal separator={,},
    },
    every axis x label/.style={
        at={(ticklabel* cs:1.00)},
        anchor=west,
    },
    every axis y label/.style={
        at={(ticklabel* cs:1.00)},
        anchor=south,
    },
}
\hypersetup{
    pdfpagelabels=true,
    plainpages=false,
    pdfauthor={Søren Bøgeskov Nørgaard, Henrik Aarup Vesterager, Lasse Thomsen},
    pdftitle={Tunable Antennas for Handsets Supporting MIMO},
    pdfsubject={},
    bookmarksdepth=3,
    bookmarksnumbered=true,
    colorlinks,
    citecolor=black,
    filecolor=black,
    linkcolor=black,
    urlcolor=black,
    pdfstartview=FitH
}
\usepackage{memhfixc}
\urlstyle{sf}
\sisetup{
    output-decimal-marker = {.},
    output-exponent-marker=\textsc{e}, 
    % round-mode=places,
    output-complex-root=j, 
    % unit-color=blue,
    % scientific-notation = engineering
}
\DeclareSIUnit[number-unit-product = \,]{\permil}{\textperthousand}
\newcolumntype{R}{>{\raggedleft\arraybackslash}X}
\newcolumntype{C}{>{\centering\arraybackslash}X}
\DeclareMathAlphabet{\mathcal}{OMS}{cmsy}{b}{n}

%   Page Setup
\usepackage[top=19mm, bottom=43mm, left=12.925mm, right=12.925mm, a4paper]{geometry}
\setlength{\headheight}{2.5em}  % For fancyhdr
\setlength{\parskip}{0.5em}
\setlength{\parindent}{1.5em}
\renewcommand{\topfraction}{0.85}  % Figures
\renewcommand{\textfraction}{0.1}
\renewcommand{\floatpagefraction}{0.85}
\setcounter{tocdepth}{1}
\setcounter{secnumdepth}{2}

\titlespacing{\chapter}{0em}{0em}{*0}
\titlespacing{\section}{0em}{1em}{*0}
\titlespacing{\subsection}{0pt}{1em}{*0}
\titlespacing{\subsubsection}{0pt}{1em}{*0}
% \titleformat{name=\part}[display] {}{}{0em}{\fontfamily{phv}\fontsize{40}{48}\selectfont\centering\bfseries\thepart\\}
% \titleformat{name=\chapter}[display] {}{}{0em}{\fontfamily{phv}\selectfont\huge\thechapter\ }
% \titleformat{name=\chapter,numberless}[display] {}{}{0em}{\fontfamily{phv}\selectfont\huge}
% \titleformat{name=\section}[display] {}{}{0em}{\fontfamily{phv}\selectfont\bfseries\Large\thesection\ }
% \titleformat{name=\section}[display] {}{}{0em}{\bfseries\Large\thesection\ }
% \titleformat{name=\subsection}[display] {}{}{0em}{\fontfamily{phv}\selectfont\Large\thesubsection\ }
% \titleformat{name=\subsubsection}[display] {}{}{0em}{\normalsize\bfseries}
% \titleformat{name=\paragraph}[runin] {}{}{0em}{\normalsize\itshape}

%   Header and Items
\usepackage{fancyhdr}
\renewcommand{\headrulewidth}{0pt}
\pagestyle{fancy}
\fancyfoot{}{}{}
\fancyhead[LO]{\normalsize\rightmark}
\fancyhead[RE]{\normalsize\leftmark}
\fancyhead[LE,RO]{\normalsize\thepage}
\let\tempone\itemize
\let\temptwo\enditemize
\renewenvironment{itemize}{%
    \tempone%
    \setlength{\parskip}{0em}%
    \setlength{\itemsep}{0.25em}%
    }{\temptwo}
\let\tempthree\enumerate
\let\tempfour\endenumerate
\renewenvironment{enumerate}{%
    \tempthree%
    \setlength{\parskip}{0em}%
    \setlength{\itemsep}{0.25em}%
    }{\tempfour}
\let\tempfive\description
\let\tempsix\enddescription
\renewenvironment{description}{%
    \tempfive%
    \setlength{\parskip}{0em}%
    \setlength{\itemsep}{0.25em}%
    }{\tempsix}
\renewcommand{\labelenumi}{\arabic{enumi}.}
\renewcommand{\labelenumii}{\arabic{enumi}.~\arabic{enumii}}
\renewcommand{\labelenumiii}{\arabic{enumi}.~\arabic{enumii}.~\arabic{enumiii}}
\graphicspath{{./img/}}

%   Packages (must be loaded last)
\usepackage{csvsimple}
\usepackage[american]{circuitikz}

% \usepackage{set/showlabels,rotating}
% \renewcommand{\showlabelsetlabel}[1]%
% {\begin{turn}{80}\showlabelfont\tiny #1\end{turn}}

% COOL LOOKING FONT!!!
% \usepackage{pdfrender}
% \pdfrender{StrokeColor=black,TextRenderingMode=2,LineWidth=0.2pt}

