\noindent
The internal antennas of today's smartphones is placed along the edge of the phone due to limited space and strict size requirements. As a result the antennas are in very close proximity to the user. This results in detuning of the antenna and power absorption by the user. As the demand for higher data rates and bandwidth keeps increasing it is desirable to counteract the detuning.     
This project will investigate the development of digitally tunable LTE antennas, with minimized ground clearance, supporting MIMO.
The solution proposes the use of a digitally controllable tuner in the
matching network.

To investigate the tunable performance and the user effect interaction, three prototype designs have been simulated and measured. Three user effect cases have been simulated for each prototype and generally, the antennas shows detuning, as an effect of the user interaction. 
The prototype results have been compared and the best performing antenna design has been tested on the WiSpry PCB with a WS1040 tuner for each antenna.
A ground clearance simulation has been carried out and an antenna design with \SI{5}{mm} ground clearance has been measured on the WiSpry PCB.  

Moving the antenna designs from the prototype PCB to the WiSpry PCB introduced some high band coverage problems. The antenna and the transmission line have been modified to counteract the high band problems.  

A MIMO tunable antenna solution have been presented with a minimized ground clearance of \SI{5}{mm}. The results show promising and comparable performance with state-of-the-art antenna designs with much higher ground clearance.    

%%% Local Variables:
%%% mode: latex
%%% TeX-master: "../master"
%%% End:
