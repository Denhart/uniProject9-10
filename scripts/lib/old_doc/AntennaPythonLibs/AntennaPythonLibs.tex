\documentclass[10pt]{article}
\usepackage[utf8]{inputenc}
\usepackage{mathptmx}
\usepackage{tabularx}
\usepackage{fancyvrb}
\usepackage{graphicx}
\usepackage{amsmath}
\usepackage{siunitx}
\usepackage{subcaption}
\usepackage[top=19mm, bottom=43mm, left=12.925mm, right=12.925mm, a4paper]{geometry}
\usepackage{hyperref}
\usepackage{xcolor}
\hypersetup{
    pdfpagelabels=true,
    plainpages=false,
    pdfauthor={Søren Bøgeskov Nørgaard, Henrik Aarup Vesterager, Lasse Thomsen},
    pdftitle={Antenna Python Libraries},
    pdfsubject={},
    bookmarksdepth=3,
    bookmarksnumbered=true,
    colorlinks,
    citecolor=black,
    filecolor=black,
    linkcolor=black,
    urlcolor=black,
    pdfstartview=FitH
}

\title{Antenna Python Libraries}
\author{Søren B.\ Nørgaard, Henrik A.\ Vesterager, Lasse Thomsen}
\begin{document}
\maketitle

\begin{abstract}
    This collection of python 3 libraries makes it possible to extract data and calibrate measurements using trx-files exported directly from Satimo Passive Measurement. These can be compared to exported CST data.
\end{abstract}

\tableofcontents


\section{Files}
\begin{tabularx}{\linewidth}{lX}
    \texttt{cst.py} & Import CST files for manipulation/comparison. \\
    \texttt{l3d.py} & General 3D functionality (plotting, surface integration). \\
    \texttt{satimo.py} & Manipulation of data exported from Satimo Passive Measurement (SPM). 
\end{tabularx}

\section{Installation}
\begin{enumerate}
\item Make sure to install Python 3, numpy, scipy, and matplotlib.
\item Put the library files in a central directory (e.g.\ \texttt{C:/PathTo/Something}).
\item Add this path to the environment variable \texttt{PYTHONPATH} (\emph{Environment Variables} in Windows and \texttt{.bashrc} or \texttt{.profile} in Linux/OSX). E.g.\
    \begin{verbatim}
# ~/.profile
export PYTHONPATH=$PYTHONPATH:/home/soren/hdd/svn/project9-10/scripts/lib
    \end{verbatim}
\end{enumerate}


\section{Data Format}
The basic data format for a measurement is a $\theta \times \phi$-matrix for each frequency.
\begin{equation}
    M = \begin{bmatrix}
        m_{1,1} & m_{1,2} & \dots & m_{1,n} \\
        m_{2,1} & m_{2,2} & \dots & m_{2,n} \\
        \vdots & \vdots & & \vdots \\
        m_{m,1} & m_{m,2} & \dots & m_{m,n}
    \end{bmatrix}
\end{equation}
where $m$ is the number of $\theta$-elements and $n$ is the number of $\phi$-elements. The angles matrix elements correspond to the following angles:
\begin{table}[htbp]
    \centering
    \begin{tabular}{|l|c|c|}
        \hline
        Elements & $\theta$ & $\phi$ \\
        \hline
        $m_{1,1}$ & \ang{0} & \ang{0} \\
        $m_{1,n}$ & \ang{0} & $\ang{360}$ \\
        $m_{m,1}$ & \ang{180}$^{\dagger}$ & \ang{0} \\
        \hline
    \end{tabular}
    \caption{Format of $\theta\times\phi$-matrix. $^{\dagger}$For Satimo measurements, this is $180-22.5=\ang{157.5}$ because of the blind spot in the bottom.}
    \label{tab:matrixformat}
\end{table}

\section{Modules}
\subsection{AAU Plot}
\subsubsection{efficiency(f, e, c="-", label="")}
Plot an efficiency graph.

\begin{verbatim}
- f: Frequency axis.
- e: Efficiency (. or dB).
- c: Color/linetype string (e.g. '--b' for dashed blue).
- label: Label for the graph's legend.
\end{verbatim}

\subsubsection{end\_efficiency(f, loc=1, fontsize=8)}
Finish the efficiency plot with legend, etc.

\begin{verbatim}
- f: Frequency axis.
\end{verbatim}

\subsubsection{figure(**kwargs)}
Set up a figure of the correct dimensions and the correct font for the
report.

\begin{verbatim}
- kwargs: All arguments are passed onto the matplotlib.pyplot.figure()
        function.
\end{verbatim}

\subsubsection{freqscale(f)}
Scale to get frequency axis to MHz

\begin{verbatim}
- f: Frequency axis.
\end{verbatim}

\subsubsection{legend(**kwargs)}
Add a legend to the plot. The labels for the legend are specified with the
``label'' option when plotting a graph.

\begin{verbatim}
- kwargs: All parameters are passed on to matplotlib.pyplot.legend().
\end{verbatim}

\subsubsection{sparam(f, s, label="")}
Plot an S-parameter.

\begin{verbatim}
- f: Frequency axis for the plot.
- s: S-parameter (abs-value in dB) to plot.
- label: Label for the legend
\end{verbatim}

\subsubsection{to\_db(x)}
Convert/preserve data in dB

\begin{verbatim}
- x: Data to convert (e.g. efficiency).

Return:
Data in dB.
\end{verbatim}



\clearpage
\section{Examples}
The file structure for these examples is as shown in Table~\ref{tab:example}. To run an example, e.g.\ \texttt{example1.py}, open the desired directory in a command prompt and type 
\begin{verbatim}
python example1.py
\end{verbatim}
\begin{table}[htbp]
    \centering
    \begin{tabular}{|l|l|}
        \hline
        File & Description \\
        \hline
        \texttt{example1.py}          & Efficiency example script. \\
        \texttt{example2.py}          & Plot 3D example script. \\
        \texttt{example3.py}          & CST farfield example script. \\
        \texttt{example4.py}          & Export efficiency as a text file example script. \\
        \texttt{example5.py}          & Plotting for IEEEtran example script.  \\
        \hline
        \texttt{cst\_farfield.txt}    & Farfield exported from CST. \\
        \texttt{antenna\_meas.trx}    & Antenna measurement (designed for 2400\,MHz). \\
        \hline
        \texttt{calib/calib2450.trx}  & Calibration measurement for 2450\,MHz dipole. \\
        \hline
        \texttt{calib/HomeRef500.ref} & Reference file for 500\,MHz dipole. \\
        \texttt{calib/HomeRef600.ref} & Reference file for 600\,MHz dipole. \\
        \texttt{calib/SD740-70.ref}   & Reference file for 740\,MHz dipole. \\
        \texttt{calib/SD850-02.ref}   & Reference file for 850\,MHz dipole. \\
        \texttt{calib/SD900-51.ref}   & Reference file for 900\,MHz dipole. \\
        \texttt{calib/SD1800-45.ref}  & Reference file for 1800\,MHz dipole. \\
        \texttt{calib/SD1900-49.ref}  & Reference file for 1900\,MHz dipole. \\
        \texttt{calib/SD2050-36.ref}  & Reference file for 2050\,MHz dipole. \\
        \texttt{calib/SD2450-43.ref}  & Reference file for 2450\,MHz dipole. \\
        \texttt{calib/SD2600-28.ref}  & Reference file for 2600\,MHz dipole. \\
        \hline
    \end{tabular}
    \caption{File structure for the example.}
    \label{tab:example}
\end{table}

\clearpage
\subsection{Extract Efficiency From Satimo}

The efficiency is extracted and plotted using the following script. The output is shown in Figure~\ref{fig:example1}.
\VerbatimInput{examples/example1.py}

\begin{figure}[htbp]
    \centering
    \includegraphics[scale=0.5]{examples/ex1_efficiency.pdf} 
    \caption{Efficiency plot from the example.}
    \label{fig:example1}
\end{figure}

\clearpage
\subsection{Plot 3D Farfield}

The (rough) farfield directly exported from Satimo's 15 probes can be plotted in 3D as shown below. The result is shown in Figure~\ref{fig:example2}. A similar plot -- a 2D color plot -- is also plotted.
\VerbatimInput{examples/example2.py}

\begin{figure}[htbp]
    \centering
    \begin{subfigure}{0.49\linewidth}
        \centering
        \includegraphics[scale=0.5]{examples/ex2_3dfarfield.pdf}
        \caption{3D.}
    \end{subfigure}
    \hfill
    \begin{subfigure}{0.49\linewidth}
        \centering
        \includegraphics[scale=0.5]{examples/ex2_2dfarfield.pdf}
        \caption{2D.}
    \end{subfigure}
    \caption{Farfield plots.}
    \label{fig:example2}
\end{figure}

\clearpage
\subsection{Import a Farfield From CST}
The following example will import and plot a farfield from CST. The farfield should be exported under the Post Processing tab. The output is shown in Figure~\ref{fig:example3}.
\VerbatimInput{examples/example3.py}


\begin{figure}[htbp]
    \centering
    \begin{subfigure}{0.49\linewidth}
        \centering
        \includegraphics[scale=0.5]{examples/ex3_3dfarfield.pdf}
        \caption{3D.}
    \end{subfigure}
    \hfill
    \begin{subfigure}{0.49\linewidth}
        \centering
        \includegraphics[scale=0.5]{examples/ex3_2dfarfield.pdf}
        \caption{2D.}
    \end{subfigure}
    \caption{Farfield plots.}
    \label{fig:example3}
\end{figure}

\clearpage
\subsection{Export Efficiency for Further Analysis}

Using python and numpy, analysis and comparison can be done just as easily as in MATLAB. However, if one is more familiar with MATLAB or need the data for other software, it is advantageous to export the data. Here is a simple example of exporting the efficiency as a data file.
\VerbatimInput{examples/example4.py}

The output is a tab-separated file (\texttt{ex4\_efficiency.txt}) which can easily be imported into MATLAB or similar software.


\clearpage
\subsection{Plots for IEEEtran Articles}
It is often desirable to export graphs in a format that is compliant with the text used articles. In the standard IEEEtran format, a \emph{Times} font is used. For figures, the default font size is 8\,pt. The column width is \SI{3.5}{inches}. In the example, the figure size is set to $\SI{3.5}{inches}\times \SI{3}{inches}$.

Note that math symbols can easily be used in figure text (e.g.\ labels) in the same way as symbols are written in \LaTeX, e.g.\ ``\verb|$\theta$ in degrees|'' becomes ``$\theta$ in degrees''.

The result is shown in Figure~\ref{fig:example5}.

\VerbatimInput{examples/example5.py}

\begin{figure}[htbp]
    \centering
    \includegraphics{examples/ex5_efficiency.pdf}
    \caption{Efficiency plot in IEEEtran format.}
    \label{fig:example5}
\end{figure}

\end{document}
