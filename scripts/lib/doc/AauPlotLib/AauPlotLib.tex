\documentclass[10pt]{article}
\usepackage[utf8]{inputenc}
\usepackage{mathptmx}
\usepackage{tabularx}
\usepackage{fancyvrb}
\usepackage{graphicx}
\usepackage{amsmath}
\usepackage{siunitx}
\usepackage{subcaption}
\usepackage[top=19mm, bottom=43mm, left=12.925mm, right=12.925mm, a4paper]{geometry}
\usepackage{hyperref}
\usepackage{xcolor}
\hypersetup{
    pdfpagelabels=true,
    plainpages=false,
    pdfauthor={Søren Bøgeskov Nørgaard, Henrik Aarup Vesterager, Lasse Thomsen},
    pdftitle={AAU Plot Library},
    pdfsubject={},
    bookmarksdepth=3,
    bookmarksnumbered=true,
    colorlinks,
    citecolor=black,
    filecolor=black,
    linkcolor=black,
    urlcolor=black,
    pdfstartview=FitH
}

\title{AAU Plot Library}
\author{Søren B.\ Nørgaard, Henrik A.\ Vesterager, Lasse Thomsen}
\begin{document}
\maketitle

\begin{abstract}
    The functions of this module makes it easy to plot consistent figures for the report.
\end{abstract}

\tableofcontents


\section{Files}
\begin{tabularx}{\linewidth}{lX}
    \texttt{aauplot.py} & Library file. \\
\end{tabularx}

\section{Installation}
\begin{enumerate}
\item Make sure to install Python 3, numpy, scipy, and matplotlib.
\item Put the library files in a central directory (e.g.\ \texttt{C:/PathTo/Something}).
\item Add this path to the environment variable \texttt{PYTHONPATH} (\emph{Environment Variables} in Windows and \texttt{.bashrc} or \texttt{.profile} in Linux/OSX). E.g.\
    \begin{verbatim}
# ~/.profile
export PYTHONPATH=$PYTHONPATH:/home/soren/hdd/svn/project9-10/scripts/lib
    \end{verbatim}
\end{enumerate}


\section{Modules}
\subsection{AAU Plot}
\subsubsection{efficiency(f, e, c="-", label="")}
Plot an efficiency graph.

\begin{verbatim}
- f: Frequency axis.
- e: Efficiency (. or dB).
- c: Color/linetype string (e.g. '--b' for dashed blue).
- label: Label for the graph's legend.
\end{verbatim}

\subsubsection{end\_efficiency(f, loc=1, fontsize=8)}
Finish the efficiency plot with legend, etc.

\begin{verbatim}
- f: Frequency axis.
\end{verbatim}

\subsubsection{figure(**kwargs)}
Set up a figure of the correct dimensions and the correct font for the
report.

\begin{verbatim}
- kwargs: All arguments are passed onto the matplotlib.pyplot.figure()
        function.
\end{verbatim}

\subsubsection{freqscale(f)}
Scale to get frequency axis to MHz

\begin{verbatim}
- f: Frequency axis.
\end{verbatim}

\subsubsection{legend(**kwargs)}
Add a legend to the plot. The labels for the legend are specified with the
``label'' option when plotting a graph.

\begin{verbatim}
- kwargs: All parameters are passed on to matplotlib.pyplot.legend().
\end{verbatim}

\subsubsection{sparam(f, s, label="")}
Plot an S-parameter.

\begin{verbatim}
- f: Frequency axis for the plot.
- s: S-parameter (abs-value in dB) to plot.
- label: Label for the legend
\end{verbatim}

\subsubsection{to\_db(x)}
Convert/preserve data in dB

\begin{verbatim}
- x: Data to convert (e.g. efficiency).

Return:
Data in dB.
\end{verbatim}



\clearpage
\section{Examples}

\subsection{Plot S-Parameters}

S-parameters can be easily plotted using a script like the one below. The result is shown in Figure~\ref{fig:example1}.
\VerbatimInput{examples/ex1.py}

\begin{figure}[htbp]
    \centering
    \includegraphics{examples/ex1_sparams.pdf}
    \caption{Efficiency plot from the example.}
    \label{fig:example1}
\end{figure}

\end{document}
