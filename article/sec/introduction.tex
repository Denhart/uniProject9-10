\section{Introduction}
\label{sec:introduction}

%% Motivation
\IEEEPARstart{I}{n} today's modern society, mobile phones are not only used for calls and messages. Since the third generation of mobile phone systems, accessing the Internet and exchanging data has become main use cases in the mobile systems. As more and more devices need mobile access, and a higher throughput is desired, the fourth generation aims to improve the obtainable data rates and get a more seamless interaction with wired and wireless IP systems \cite{tanenbaum2012computer}.

The fourth generation mobile technology, the Long Term Evolution (LTE),  supports Multiple Input Multiple Output (MIMO) as a means of improving data rates. MIMO takes advantage of the scattering environment in which the phones are used to make parallel streaming of data, through different physical routes, possible.


% Correlation
To enable MIMO, a minimum of two antennas must be present in the mobile handset. To gain the most from MIMO, the correlation between the farfield patterns of the two antennas must be as low as possible. Correlation can be decreased by placing the antennas far apart with respect to a wavelength. However, at the low frequency bands of LTE -- around \SI{700}{MHz} -- the free-space wavelength is around \SI{428}{mm} which is far larger than a mobile phone. Therefore, increasing the distance is not the answer, and the antennas must instead be designed to have their farfield patterns pointing in different directions, have different polarizations, etc.

% Size constraint

% User effects + Size constraint
Apart from the antennas being decorrelated, the total efficiency of each antenna must be high to increase performance. As modern smart-phones are very screen-oriented, having the screen take up most of the from of the phone, only little room is left for antennas near the edge of the phone. In practice, when a user is holding the phone, the total efficiency drops. This is both due to power absorption and the user detuning the resonance of the antenna, increasing the mismatch loss \cite{Samantha2014UserEff}. Some absorption loss can be beneficial in decorrelating the two MIMO antenna patterns \cite{Samantha2014UserEff} while the mismatch loss is undesirable.

% Tunable


%% State of the art

%% Overview
In this project,
