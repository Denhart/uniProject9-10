\section{Introduction}
\label{cha:intro}

%% Motivation
\IEEEPARstart{T}{he} mobile phone era began to take its modern form in the eighties when the development of analog cell-based mobile phone systems began. The AMPS systems, developed by Bell Labs and installed in the United States in 1982, being the most successful of the first generation mobile phone systems, proved that mobile personal telephony was here to stay \cite{tanenbaum2012computer}. In the second generation, speech was transmitted in digital form instead of analog. The GSM system from Europe became the de-facto world wide standard for 2G, while still only being aimed at voice communication \cite{tanenbaum2012computer}. Towards the third generation, and with the transition towards smartphones, mobile phones were no longer only a means of communicating voice and short messages, but also a means of accessing the Internet and exchanging data. The increase in music and video services online made the demand for higher data rates larger and larger while the number of users increased as well.

% LTE: MIMO antennas for higher performance: Requires low correlation + high efficiency
The technology for the fourth generation is the Long Term Evolution (LTE). The fourth generation of mobile phone systems aimed to increase the data rates as well as getting a more seamless interaction with wired and wireless IP networks \cite{tanenbaum2012computer}. A way of getting there is the use of MIMO, i.e.\ having multiple antennas in the mobile phones, making it possible to either communicate data in parallel, increasing the data rate, or to increase the signal strength using diversity techniques. For this reason, the LTE specification supports multiple antennas to be integrated for LTE MIMO and diversity use \cite{holma2011lte}.

In order to achieve good MIMO and diversity performance, the correlation between the radiation patterns of the LTE antennas should be low, i.e.\ the signals received by the two should be as different as possible in order to gain the most \cite{Tim2012Practical}. Patterns can be decorrelated by spacing the antennas far apart compared to the wavelength. However, this is not practical in a mobile phone at the low band frequencies as the free-space wavelength at \SI{700}{MHz} is \SI{429}{mm}. The correlation can be improved by having the antennas polarized differently but this is no easy task.

% Lower frequency bands may be deployed (cite: Samantha2015tunableAntennas)
The demand for bandwidth increases as more and more users desire a greater throughput, new bands are being licensed. Part of the spectrum around \SI{600}{MHz}, previously used for television broadcasting, is being considered for extending the LTE bands \cite{Samantha2015tunableAntennas}. While the lower frequencies provide great penetration for long-range communication, the wavelength, and hence the antennas, tend to either increase in size or decrease the efficiency or bandwidth \cite{hilbert2015tradeoff}. 

% Screen-dominant: Antennas near edge
% Less space in phones -> smaller -> less BW/higher Q (cite: hilbert2015tradeoff)
In modern mobile phones, the screen is the dominant user interface, taking up almost all of the front-side of the phone. This means that only little space, along the edge of the phone, is available for antennas. As described in \cite{hilbert2015tradeoff}, the smaller available area means that either the efficiency or the bandwidth must decrease.

% Phones in-use: Detuning cite: pelosi2009grip
As the antennas are placed close to the edge of the phone, they are in very close proximity to the user. This has the effect of absorbing part of the power and also detuning the antenna \cite{pelosi2009grip}. The antennas can be designed to have resonances which are determined by a matching circuit placed immediately before the antenna but as the antenna changes its resonance based on whether or not a user is present, the matching circuitry would need to be variable to account for different use cases.

% Solution: Lower BW, Tunable matching network
The solution to the low available bandwidth and the detuning caused by the user, is to use a digitally controllable tuner in the matching network. This makes it possible to have a lower bandwidth, covering a minimum of only the largest LTE band -- not all at the same time -- and then re-tune the resonance to the desired band. This way, all bands could be covered at a decent efficiency and the loading caused by the user could be minimized by counter-tuning the antenna based on the user's behavior.

Several solutions exist for digitally tunable capacitors. Varactor diodes can be used as tunable capacitors by altering the bias voltage while CMOS tunable capacitors consist of banks of capacitors which are switched in and out of circuit using CMOS technology. MEMS tunable capacitors come in two variants where one uses MEMS switches to switch capacitors like the CMOS tuners. The other variant alters the proximity of two parallel plates, thereby changing the capacity \cite{gu2014rf}.

% State-of-the-art
Previous antenna designers have dealt with developing tunable antennas for the LTE bands supporting MIMO. Using MEMS tunable capacitors, \cite{ilvonen2014multiband} managed to design a quite efficient design with two antennas. The ground clearance for this design is \SI{15}{mm} and may be closing in on the size constraints for practical implementation in a phone. In \cite{morris2014tunable}, a single antenna was designed using a MEMS tuner. While being small and rather efficient, this design only consisted of a single antenna and does, for this reason, not support MIMO for LTE. In \cite{xia2015compact}, a CMOS tuner was used and a very compact and efficient design has been developed while still only for a single antenna. A MEMS tuner was used for tuning the side antenna in \cite{tatomirescu2015alternative} while the top antenna was fixed and showed good results. This design, however, did only cover the low bands below \SI{960}{MHz}. Finally, in \cite{trinh2016reconfigurable}, a design for reaching towards 5G was designed. Showing very good results, the design only consisted of a single antenna.

In this project, a dual resonance antenna design for LTE supporting MIMO will be designed. The goal is to minimize the ground clearance so the antenna would be attractive in a practical design. The aim is to investigate how ground clearance affects the bandwidth and design a MIMO antennas system using the smallest practical clearance, covering all LTE bands from \SI{700}{MHz} to \SI{960}{MHz} and from \SI{1710}{MHz} to \SI{2650}{MHz}.

%% Overview
% Report overview
The first part of the report -- Chapter~\ref{cha:problem_analysis} -- contains a problem analysis. The goal of this chapter is to cover all the theory and background knowledge needed in order to successfully set up requirements and design the final product. Everything from basic antenna parameters to the background of LTE and measurement techniques will be described in this chapter. 
In Chapter~\ref{cha:reqspec}, all functional and specific requirements for the antenna design will be summarized in the requirement specification. The requirements define the frequency bands of interest as well as measures of bandwidth, etc. 
Chapter~\ref{cha:testspec} -- the test specification -- describes how the specific requirements from the requirement specification will be tested.
After the requirements and test procedures have been defined, the product development will begin from Chapter~\ref{cha:nousersim}. Here, three preliminary antenna designs will be developed and simulated in free-space. 
In Chapter~\ref{cha:usereff}, the preliminary designs will be simulated in three different use cases: Data mode, play mode, and talk mode, to observe the effect of a user holding the phone. Here, the Specific Absorption Rate (SAR) will also be simulated in order to ensure compliance with the requirements for this.
The preliminary designs will be prototyped and measured in Chapter~\ref{cha:prototypes} with discrete components for the matching network and tuner. The most promising of these designs will later be used on a PCB with two MEMS tuners.
A smaller design, more suited for practical implementation in a phone, will be developed in Chapter~\ref{cha_intro_5mm}. The antenna will be simulated and measured.
In Chapter~\ref{cha:pcb}, the most promising design from Chapter~\ref{cha:prototypes} as well as a modified version of the design from Chapter~\ref{cha_intro_5mm} will be moved to a PCB with a MEMS tuner for each antenna. The designs will be modified to fit the new board and a sweep measurement of the $S$-parameters and the total efficiency will be carried out for each design.
Finally, in Chapter~\ref{cha:conclusion}, a conclusion will be summing up the results from the report.

In Appendix~\ref{cha:autotest}, automatic testing software developed during the project will be described. Software is developed for automatically sweeping a Vector Network Analyzer (VNA) and the measurements in an anechoic chamber. A circuit for fiber optic communication in the anechoic chamber is developed to automate adjusting of the tuner from outside the chamber.
Appendix~\ref{cha:postproc} describes the Python libraries developed for post processing data from the VNA, the anechoic chamber, and from CST Microwave Studio. The library for plotting the graphs, used in this report, is also documented here.
Lastly, Appendix~\ref{cha:cstmacro} shows a CST script for automatically sweeping and exporting the total efficiency from CST Microwave Studio.
