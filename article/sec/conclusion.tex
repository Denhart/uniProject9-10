\section{Conclusion}
\label{cha:conclusion}
This report has documented the development of digitally tunable LTE antennas supporting MIMO. 

%% Preliminary designs
Three preliminary designs, which all cover the required bands in free-space, have been designed. Each designs has its strengths and weaknesses and they all use a ground clearance of \SI{9.5}{mm} to \SI{10}{mm} for the top antenna and \SI{7}{mm} to \SI{9.5}{mm} for the side antenna. 

The three preliminary designs were also simulated, in the proximity of a user, in data mode, play mode, and talk mode. Here, the triangle-feed designs as well as the dual-feed design generally held up the best although severe detuning was apparent in every design. The Specific Absorption Rate (SAR) was also simulated and every design met the requirements of a maximum SAR of \SI{2}{W/kg}.

The simulations were simplified, not taking into account non-ideal components, component placement, and transmission lines. To get a more accurate picture of the performance, prototypes were built of the three preliminary designs. The $S$-parameters and the total efficiency were measured for a variety of discrete tuning capacitor values. The folded monopole design did not cover the low band very well and the dual-feed design was not tunable in the side antenna. For this reason, the triangle-feed design was chosen as the first design to be realized on a PCB with a MEMS tuner.

%% The effect of ground clearance -> 5mm minimized design
As the space for antennas in today's mobile phones is limited by the large screen size, a smaller design was developed to also be implemented with the tuner PCB. An investigation of the effect of lowering the ground clearance was made, and showed that a very reasonable tunable design could be realized with only \SI{5}{mm} of ground clearance -- on a par with the preliminary designs over the tunable range. The minimized design was measured with the tuner PCB but showed problems covering the high band. The design was therefore modified with an extra pair of arms before the final implementation with the tuner PCB.

The triangle-feed design, like the minimized design, showed problems covering the high end of the high band when moved to the tuner PCB. The low band, for both the top and the side antenna, as well as most of the high band for the top antenna was covered acceptably, while the side antenna was not resonating as required in the high band.

The modified minimized design, while being slightly better in the high band, still had trouble in the very high end of the spectrum. For the final design, it was observed that adding a small amount of shunt capacitance near the middle of the transmission lines, going from the SMA connectors to the antennas, improved the high band performance vastly. The resulting design showed that the top antenna was able to cover the whole low band at above \SI{-4}{dB} total efficiency and the high band above \SI{-3}{dB} from \SI{1710}{MHz} to around \SI{2500}{MHz}. From \SI{2500}{MHz} and up, the efficiency decreased. The side antenna generally showed a lower bandwidth and efficiency with a total efficiency from \SI{-10}{dB} to \SI{-4.5}{dB} in the low band and an efficiency above \SI{-3}{dB} from around \SI{1710}{MHz} to around \SI{2300}{MHz} except for a notch around \SI{2000}{MHz}. These results show that a low-ground clearance design is possible while still covering most of the LTE bands at an acceptable total efficiency.

%% Correlation high -> only MIMO above x MHz
The envelope correlation has been simulated for all designs. In order to have good MIMO performance, a low correlation -- below 0.5 -- is desired. The modified minimized design showed a correlation above 0.5 from \SI{700}{MHz} to around \SI{900}{MHz} in free-space which makes it unsuited for MIMO applications in the low band. The correlation is much lower in the high band, so MIMO could still be used at these frequencies. In the user effect simulations, the correlation in the low band generally dropped, making the MIMO useful in part of the low band as well. The high correlation at low frequencies is to be expected as a large part of the ground plane is used as a radiating element for both top and side antenna, making the radiation patterns more similar. MIMO and diversity schemes would therefore be better implemented at higher frequencies.

\newpage

%% Comparison with state-of-the-art
\section{Conparison of Reconfigurable LTE Antenna Designs}

\begin{table}[htbp]
    \centering
    \footnotesize
    \begin{tabularx}{\linewidth}{p{21mm}Xk{13mm}k{17mm}k{13mm}k{20mm}k{17mm}k{17mm}}
        \toprule
        &&&&&& \multicolumn{2}{c}{$\eta_{\text{tot}}$ of main antenna} \\\cline{7-8}
        Antenna & Note & Tuner type & Antenna volume (\si{mm\cubed}) & Antenna area (\si{mm\squared}) & Total dimensions (\si{mm\cubed}) & Low band (\si{\%}) & High band (\si{\%})  \\
        \midrule
        Monopole top                & Monopole                    & Discrete & 3885   & 555   & $130\times62\times7$     & 38--60 & 43--98 \\
        Monopole side               & Monopole                    & Discrete & 2646   & 378   & $130\times62\times7$     & 28--64 & 31--84 \\
        Triangle-feed top           & Non-resonant and microstrip & Discrete & 4340   & 620   & $140\times69\times7$     & 47--70 & 59--92 \\
        Triangle-feed side          & Non-resonant and microstrip & Discrete & 4550   & 650   & $140\times69\times7$     & 48--72 & 59--91 \\
        Non-resonant top            & Non-resonant                & Discrete & 2850   & 570   & $129.5\times67\times6.6$ &        & \\
        Non-resonant side           & Non-resonant                & Discrete & 3087.5 & 617.5 & $129.5\times67\times6.6$ &        & \\
        \midrule
        \cite{ilvonen2014multiband} main & Planar ProtoM          & MEMS     & 1170   & 900   & $120\times60\times1.5$   & 30--57 & 44–78  \\
        \cite{ilvonen2014multiband} aux  & Planar ProtoM          & MEMS     & 1170   & 900   & $120\times60\times1.5$   & 29--57 & 43--78 \\
        \cite{ilvonen2014multiband} main & ProtoM                 & MEMS     & 3900   & 900   & $120\times60\times5$     & 49--72 & 56--88 \\
        \cite{ilvonen2014multiband} aux  & ProtoM                 & MEMS     & 3900   & 900   & $120\times60\times5$     & 48--72 & 55--88 \\
        \cite{morris2014tunable}         & Monopole               & MEMS     & 1500   & ?     & ?                        & 28--60 & 45--97 \\
        \bottomrule
    \end{tabularx}
    \caption{Comparison of reconfigurable LTE antenna designs (measured free space parameters). The total efficiencies the maximum obtainable bandwidth in-band for all measured capacitor values.}
    \label{tab:comparison_reconf_lte}
\end{table}

% morris2014tunable Tunable Antennas for Mobile Devices: Achieving High Performance in Compelling Form Factors 
% http://ieeexplore.ieee.org/stamp/stamp.jsp?tp=&arnumber=6848618
% 
% A Compact Multi-band Tunable LTE Antenna for Mobile Applications
% http://ieeexplore.ieee.org/stamp/stamp.jsp?tp=&arnumber=7304964
% 
% Alternative Duplexing for LTE FDD using the Theory of Characteristic Modes
% http://ieeexplore.ieee.org/stamp/stamp.jsp?tp=&arnumber=7228959
% 
% Reconfigurable Antenna for Future Spectrum Reallocations in 5G Communications
% http://ieeexplore.ieee.org/stamp/stamp.jsp?tp=&arnumber=7347359


\fixme{Search-and-replace Broken arrow antenna. And Allan. And Mulepose}

\fixme{Finish filling out the table.}


All the measured designs from from the report are summarized in Table~\ref{tab:comparison_reconf_lte} together with the designs discussed in the introduction, Chapter~\ref{cha:intro}. Here, the range of total efficiency for each design is shown, for both the low and the high band, as well as the dimensions of the designs. 

In the low band, most of the measurements show comparable efficiencies to the state-of-the-art designs. In the high band, the prototypes a PCB with discrete components also show very comparable results. It is clear that the efficiency in the high band is severely lower on the tuner PCB. The small minimum value is due to some part of the high band not being covered while most of the band \emph{is} covered at above \SI{-3}{dB}.

The top antenna of the modified minimized design, with only \SI{5}{mm} ground clearance, is very much on a par with designs with much more ground clearance (up to \SI{2500}{MHz}) while the side antenna suffers from a few in-band notches. 

Compared to \cite{ilvonen2014multiband}, the area is smaller due to the low ground clearance, while the volume of the antenna is larger as the height is \SI{7}{mm}. A natural next step would be to try to lower the height to make the design even more compact, as done in \cite{ilvonen2014multiband}. Another next step would be to re-design the PCB to see if the high band performance can be improved. In this case, the RFFE optical interface could be incorporated into the board as well.

A \SI{5}{mm} version of the triangle-feed design may be successful if capacitance is added to the transmission line as with the modified minimized design. The \SI{10}{mm} version appeared to be well-behaved so this design might also be good for LTE if it could be minimized.

The results show that it is possible to design LTE antennas with only \SI{5}{mm} ground clearance while still having an acceptable total efficiency over the range of the tuner.
